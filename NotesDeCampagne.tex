\documentclass[10pt,a4paper]{article}
\usepackage[utf8]{inputenc}
\usepackage{amsmath}
\usepackage{amsfonts}
\usepackage{amssymb}

\begin{document}

\title{Notes de campagne}
\author{Bertolen}

\maketitle
\pagenumbering{gobble}

\newpage

\tableofcontents
\pagenumbering{roman}

\newpage

\pagenumbering{arabic}

\section{Acte 1}

\subsection{Objectifs}

L'objectif principal de cet arc est de confronter les PJ au surnaturel pour la première fois. Ainsi ils se seront premièrement témoins du surnaturel puis ils vont le voir se manifester dans eux. Jusqu'à peut être le provoquer intentionnellement.

Le deuxième objectif est de sous-ligner aux joueurs que les différences entre notre monde et le Monde des Ténèbres ne se limitent pas uniquement au surnaturel. Elles vont jusque dans la population humaine qui est plus corrompue et plus égocentrique que celle de notre propre monde.

\subsection{Cadre}

Le cadre pour cet arc est assez vague. Ce qui va le plus jouer ce sont les PNJ avec lesquels les PJ vont interagir. Avant de tomber sur Azil tous les PNJ rencontrés seront hostiles aux PJ ou alors les laisseront pour cause perdue.

\subsubsection{Scénario 1 - Ezio}

\subsubsection{Scénario 2 - Chloé}

\subsubsection{Scénario 3 - John Ahmed}

\subsubsection{Scénarios d'enquête}

\subsubsection{Scénario final}

\subsection{PNJ}

\subsubsection{Ezio}

\subsubsection{John Ahmed}

\subsubsection{Chloé Demougin}

\subsubsection{Asil Akalay, dit Père Bâtard}

\subsubsection{Aurélien Carpentier, dit Brise Gorges}

\newpage
\section{Acte 2}

\subsection{Objectifs}

\subsection{Cadre}

\subsection{Lieux Notables}

\subsection{PNJ}

\end{document}